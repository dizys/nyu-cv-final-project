% This must be in the first 5 lines to tell arXiv to use pdfLaTeX, which is strongly recommended.
\pdfoutput=1
% In particular, the hyperref package requires pdfLaTeX in order to break URLs across lines.

\documentclass[11pt]{article}

% Remove the "review" option to generate the final version.
\usepackage[]{acl}

% Standard package includes
\usepackage{natbib}
\usepackage{lastpage}
\usepackage{fancyhdr}
\usepackage{times}
\usepackage{latexsym}
\usepackage{float}
\usepackage{graphicx}

% For proper rendering and hyphenation of words containing Latin characters (including in bib files)
\usepackage[T1]{fontenc}
% For Vietnamese characters
% \usepackage[T5]{fontenc}
% See https://www.latex-project.org/help/documentation/encguide.pdf for other character sets

% This assumes your files are encoded as UTF8
\usepackage[utf8]{inputenc}

% This is not strictly necessary, and may be commented out,
% but it will improve the layout of the manuscript,
% and will typically save some space.
\usepackage{microtype}

\pagestyle{fancy}
\fancyhead{}
\renewcommand{\headrulewidth}{0pt}
\fancyfoot[C]{Page \thepage \hspace{1pt} of \pageref{LastPage}}

% If the title and author information does not fit in the area allocated, uncomment the following
%
%\setlength\titlebox{<dim>}
%
% and set <dim> to something 5cm or larger.

\title{Detecting AI-Generated Images  using ResNet}

\author{Ziyang Zeng \and Zhehu Yuan \and Yifan Jin \\
  Dept. of Computer Science \\
  New York University \\
  251 Mercer Street, New York, NY \\
  \texttt{zz2960@nyu.edu, zy2262@nyu.edu, yj2063@nyu.edu}}

\begin{document}
\maketitle
\begin{abstract}
  Abstract goes here.
\end{abstract}

\section{Introduction}

Introduction goes here.


\section{Related Work}

Related work goes here.

\section{Datasets}

In order to train a classifier that can distinguish between real and generated images, we need a labeled dataset that consists both of real and generated images for supervised learning. Datasets of this kind is not common, and we'd like to experiment on the latest state-of-the-art image generation AI models, such as DALL·E 2 and Stable Diffusion. We started with real photos from existing public datasets as our raw datasets. Then, we use Diffusers-based image-to-image model to generate images from real photos. This way, we ended up with a labeled dataset of real photos and generated photos.

\subsection{Raw Datasets}

\subsection{Diffusers}

\subsubsection{CLIP guidance}

\subsubsection{Stable Diffusion}

\subsubsection{DALL·E 2}

\section{Classifier}

Classifier goes here.

\section{Experiments}

Experiments go here.

\subsection{Experiment Settings}

Experiment Settings go here.

\subsection{Evaluation Metrics}

Evaluation metrics go here.

\section{Experiment Results}

Experiment results go here.

\section{Conclusion}

Conclusion goes here.

\section{Discussion and Future Work}

Discussion and future work go here.

% \printbibliography
\bibliographystyle{plain}
\bibliography{refs}

\end{document}
